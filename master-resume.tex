\documentclass{article}

%A Few Useful Packages
\usepackage{marvosym}
\usepackage[table,xcdraw]{xcolor}
\usepackage{tabularx} % in the preamble
\usepackage{fontspec} 					%for loading fonts
\usepackage{xunicode,xltxtra,url,parskip} 	%other packages for formatting
\RequirePackage{color,graphicx}
\usepackage[usenames,dvipsnames]{xcolor}				%better formatting of the A4 page
% an alternative to Layaureo can be ** \usepackage{fullpage} **
\usepackage{supertabular} 				%for Grades
\usepackage{titlesec}					%custom \section
\usepackage[top=5mm,left=9mm,right=9mm,bottom=0mm]{geometry}

%Setup hyperref package, and colours for links
\usepackage{hyperref}
\definecolor{linkcolour}{rgb}{0,0.2,0.6}
\hypersetup{colorlinks,breaklinks,urlcolor=linkcolour, linkcolor=linkcolour}

%FONTS
\defaultfontfeatures{Mapping=tex-text}
%\setmainfont[SmallCapsFont = Fontin SmallCaps]{Fontin}
%%% modified for Karol Kozioł for ShareLaTeX use
\setmainfont[
SmallCapsFont = Fontin-SmallCaps.otf,
BoldFont = Fontin-Bold.otf,
ItalicFont = Fontin-Italic.otf
]
{Fontin.otf}
%%%

%CV Sections inspired by: 
%http://stefano.italians.nl/archives/26
\titleformat{\section}{\Large\scshape\raggedright}{}{0em}{}[\titlerule]
\titlespacing{\section}{0pt}{3pt}{3pt}
%Tweak a bit the top margin
%\addtolength{\voffset}{-1.3cm}

%Italian hyphenation for the word: ''corporations''
\hyphenation{im-pre-se}

%-------------WATERMARK TEST [**not part of a CV**]---------------
\usepackage[absolute]{textpos}

\setlength{\TPHorizModule}{30mm}
\setlength{\TPVertModule}{\TPHorizModule}
\textblockorigin{2mm}{0.65\paperheight}
\setlength{\parindent}{0pt}

%--------------------BEGIN DOCUMENT----------------------

\begin{document}

\pagestyle{empty} % non-numbered pages

\font\fb=''[cmr10]'' %for use with \LaTeX command

%--------------------TITLE-------------
\par{\centering
		{\Huge  \textsc{Aditya Gajbhiye}
	}\par}

%--------------------SECTIONS-----------------------------------
%Section: Personal Data
\section{Personal Data}

\begin{tabular}{rlcc}
    \textsc{Department:} &$4^{th}$ Year UG , Dept. of Computer Science , IIT Kanpur\\
    \textsc{Address:}   & D-312/9 IIT,Kanpur 208016 &
    \textsc{Phone:}     & +91-7755057618\\
    \textsc{email:}     & adityagj@iitk.ac.in , aditya.gajbhiye.20@gmail.com &
    \textsc{Github Profile:}   & github.com/AdiGajbhiye
\end{tabular}

%Section: Education
\section{Educational Qualifications}
% Please add the following required packages to your document preamble:
% 
% If you use beamer only pass "xcolor=table" option, i.e. \documentclass[xcolor=table]{beamer}

\centering
\begin{tabular}{|l|l|l|l|}
\hline
\rowcolor[HTML]{C0C0C0} 
\hline
\textbf{Year} & \textbf{Degree/Certificate} & \textbf{Institute/Board} & \textbf{CPI/\%} \\ \hline
2014-Current  & B.Tech., CSE                & IIT Kanpur               & 7.4/10          \\ \hline
2014          & Maharashtra SSC                      & State Board Maharashtra  & 91\%          \\ \hline
2012          & Matriculation               & CBSE                     & 9.8/10          \\ \hline
\end{tabular}

\section{Internships}
\begin{tabularx}{\textwidth}{Xl}
    \textbf{Intern} & \textit{May 8, 2017 - July 14, 2017} \\
    \textbf{VMware} & Bangalore, India
\end{tabularx}
\begin{flushleft}
\begin{itemize}
    \item Developed Kubernetes cluster monitoring and alerting system.
    \item Improved performance by reducing workload, increased robustness, performed testing using Mockito.
    \item Learned Kubernetes architecture and vROps Adapter SDK as a part of development process.
    \item Followed Agile development methodology.
\end{itemize}
\end{flushleft}
\begin{tabularx}{\textwidth}{Xl}
    \textbf{Intern / Full-Stack Developer} & \textit{May 9, 2016 - July 8, 2016} \\
    \textbf{AxisRooms} & Bangalore, India
\end{tabularx}
\begin{flushleft}
\begin{itemize}
    \item Built a web based GUI to on-board customers which earlier was done manually with scripts. 
    \item Used angularjs as front-end, nodejs \& express as back-end and mongodb as database(MEAN Stack).
    \item Successfully increased efficiency and productivity of on-boarding process with simple interface. 
    \item Developed RESTful APIs for all CRUD operations.
\end{itemize}
\end{flushleft}
\section{Projects}
\begin{tabularx}{\textwidth}{Xl}
	      \textbf{KeyboardNinja Ruby on Rails Game}
	      & \textit{October 2016}
	      \end{tabularx}
	      \begin{itemize}
		\vspace{-2mm} \setlength\itemsep{-0.2em}
		\item An online multiplayer typing game build on Ruby on Rails.
		\item Developed RESTful APIs for creating, joining games, getting result and updating players' current position.
		\item Built and deployed using Ruby on Rails, Coffeescript, MySql, Postgres, AJAX, jQuery and Heroku. Open sourced under MIT License.
		\item Read ``Learn Web Development with Rails'' by Michael Hartl and grasped the basics of Rails.
	      \end{itemize}
\begin{tabularx}{\textwidth}{Xl}
	      \textbf{College Festival APP(Android)}
	     & \textit{May 2015 - Oct 2015}
	      \end{tabularx}
	      \begin{itemize}
	      	\vspace{-2mm} \setlength\itemsep{-0.2em}
	      	\item Built the Android application for the college cultural festival, Antaragni.
	      	\item Users can view schedule, event details, venue map and contact details of event organizer.
	      	\item Notifications, Result declaration, Reminders are provided through GCM.
	      	\item Licensed under Apache License. App is designed to be usable for other events.
	      \end{itemize}
\begin{tabularx}{\textwidth}{Xl}
	 \textbf{Hostel Room Allocation} (CS252 Project)
	 & \textit{November 2016}
	 \end{tabularx}
\begin{itemize}
\item Developed an application for allocating rooms based on students' preferences.
\item Solved using Stable marriage and Hungarian algorithms as base.
\item Developed using MEAN stack and python, including libraries like Passport (authentication) and Bcrypt (encryption).
\end{itemize}
\begin{flushleft}

	  \textbf{Perl Compiler} (CS335 Project) - Developed Perl compiler using Python.\\
	  \textbf{DBMS-UI} (CS315 Project) - Developed Qt-based GUI for performing queries on MySQL server.\\
      \textbf{De Bruijn Sequences} (CS201 Project) - Study and
understand the creation of de bruijn sequences , their real life
applications and using them to do mathematics based card tricks.
\end{flushleft}

\section{Technical skills}
\begin{itemize}
    \item  Programming Languages: C | C++ | JAVA | Python | Ruby | Perl
    \item  Web Development: HTML | CSS | PHP | JavaScript | Typescript | AngularJS | NodeJS | Express | BootStrap | MongoDB | MySQL
    \item  Technologies: Kubernetes | Docker | Tensorflow | Git | Perforce | Android Studio | vROps Adapter SDK
    \item  Operating Systems/Tools: Linux (Arch Linux \& Ubuntu) | Windows | Sublime | Visual Studio Code | Vim | Latex | Eclipse
\end{itemize}

\section{Relevent Courses}
\centering{}
\begin{tabular}{l|l|l}
CS-771 Machine Learning * & CS-425 Computer Networks * & CS-628 Computer Security \\
CS-315 Principles of Database Systems  & CS-210 Data Structures and Algorithms & CS-345 Design and Analysis of Algorithm\\
CS-330 Operating Systems & CS-335 Compiler Design & CS-340 Theory of Computation\\
MTH-101 Analytical Calculus & MTH-102 Linear Algebra and DE & MSO-201 Probability and Statistics \\
CS-201 Discrete Mathematics & CS-202 Abstract Algebra & CS-203 Logic \\
ESC-101 Introduction to Programming & ESC-201 Introduction to Electronics & CS-220 Computer Organization \\
MSO-202 Complex Variables * & MTH-426 Mathematical Modelling & MTH-428 Mathematical Methods *\\
\end{tabular}
\begin{flushright}
\textit{* - ongoing}
\end{flushright}

\end{document}
